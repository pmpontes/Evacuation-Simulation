%----------------------------------------------------------------------------------------
%	PACKAGES AND OTHER DOCUMENT CONFIGURATIONS
%----------------------------------------------------------------------------------------

\documentclass[12pt]{article}

\usepackage[utf8]{inputenc}
\usepackage[T1]{fontenc}
\usepackage{lmodern}
\usepackage{parselines}
\usepackage[portuguese]{babel}
\usepackage{graphicx}
\usepackage[document]{ragged2e}
\usepackage{listings}
\usepackage{xcolor}
\usepackage{geometry}
\geometry{
	a4paper,
	total={170mm,257mm},
	left=30mm,
	right=30mm,
	top=30mm,
	bottom=30mm,
}
\usepackage{amsmath}
\usepackage{hyperref}
\usepackage{url}
\usepackage{float}
\usepackage{tabularx}
\usepackage{booktabs}
\usepackage{indentfirst}
\usepackage{colortbl}
\usepackage{caption} 
\captionsetup[table]{skip=10pt}

\graphicspath{ {images/} }

\colorlet{punct}{red!60!black}
\definecolor{background}{HTML}{EEEEEE}
\definecolor{delim}{RGB}{20,105,176}
\colorlet{numb}{magenta!60!black}


\lstdefinelanguage{json}{
    basicstyle=\normalfont\ttfamily,
    numbers=left,
    numberstyle=\scriptsize,
    stepnumber=1,
    numbersep=8pt,
    showstringspaces=false,
    breaklines=true,
    frame=lines,
    backgroundcolor=\color{background},
    literate=
     *{0}{{{\color{numb}0}}}{1}
      {1}{{{\color{numb}1}}}{1}
      {2}{{{\color{numb}2}}}{1}
      {3}{{{\color{numb}3}}}{1}
      {4}{{{\color{numb}4}}}{1}
      {5}{{{\color{numb}5}}}{1}
      {6}{{{\color{numb}6}}}{1}
      {7}{{{\color{numb}7}}}{1}
      {8}{{{\color{numb}8}}}{1}
      {9}{{{\color{numb}9}}}{1}
      {:}{{{\color{punct}{:}}}}{1}
      {,}{{{\color{punct}{,}}}}{1}
      {\{}{{{\color{delim}{\{}}}}{1}
      {\}}{{{\color{delim}{\}}}}}{1}
      {[}{{{\color{delim}{[}}}}{1}
      {]}{{{\color{delim}{]}}}}{1},
}

\definecolor{dkgreen}{rgb}{0,0.6,0}
\definecolor{gray}{rgb}{0.5,0.5,0.5}
\definecolor{mauve}{rgb}{0.58,0,0.82}

\lstset{frame=tb,
  language=Java,
  aboveskip=3mm,
  belowskip=3mm,
  showstringspaces=false,
  columns=flexible,
  basicstyle={\small\ttfamily},
  numbers=none,
  numberstyle=\tiny\color{gray},
  keywordstyle=\color{blue},
  commentstyle=\color{dkgreen},
  stringstyle=\color{mauve},
  breaklines=true,
  breakatwhitespace=true,
  tabsize=3
}

\hypersetup{
  colorlinks, linkcolor=black
}

\begin{document}

\begin{titlepage}

\newcommand{\HRule}{\rule{\linewidth}{1mm}} % Defines a new command for the horizontal lines, change thickness here

\center % Center everything on the page
 
%----------------------------------------------------------------------------------------
%	HEADING SECTIONS
%----------------------------------------------------------------------------------------

\includegraphics{feup.jpg}

\textsc{\large Agentes e Inteligência Artificial Distribuída}\\[0.8cm] % Major heading such as course name
\textsc{\large 4º ano do Mestrado Integrado em Engenharia Informática e Computação}\\[0.8cm] % Minor heading such as course title

%----------------------------------------------------------------------------------------
%	TITLE SECTION
%----------------------------------------------------------------------------------------

\HRule \\[1.2cm]
{ \huge \bfseries \textit{Simulação de Evacuação com Agentes}}\\[0.6cm] % Title of your document
{ \large \bfseries Relatório Intercalar} \\[0.6cm]
\HRule \\[2cm]
 
%----------------------------------------------------------------------------------------
%	AUTHOR SECTION
%----------------------------------------------------------------------------------------


% If you don't want a supervisor, uncomment the two lines below and remove the section above
\Large \emph{Authors:}\\[0.5cm] \normalsize
Gil \textsc{Domingues}\\[0.1cm]  
- up201304646@fe.up.pt\\[0.1cm]
Pedro \textsc{Pontes}\\[0.1cm]
- up201305367@fe.up.pt\\[2cm] 

%----------------------------------------------------------------------------------------
%	DATE SECTION
%----------------------------------------------------------------------------------------

{\large \today}\\[0cm] % Date, change the \today to a set date if you want to be precise

%----------------------------------------------------------------------------------------
%	TABLE OF CONTENTS & LISTS OF FIGURES AND TABLES
%----------------------------------------------------------------------------------------

\clearpage 

\tableofcontents

\clearpage 
%----------------------------------------------------------------------------------------
%	INTRODUÇÃO
%----------------------------------------------------------------------------------------
\justify\normalsize

\section{Introdução} 

Uma evacuação implica mover pessoas de um dado local devido à ocorrência de uma situação de (potencial) catástrofe. Exemplos incluem a evacuação de um edifício em chamas ou de uma localidade, antes, durante ou após um desastre natural, como uma cheia ou terramoto. Tipicamente, de uma evacuação acabam por resultar feridos ou mesmo mortes, vítimas de espezinhamento. 

Com o aumento da frequência de situações que implicam a evacuação de um elevado número de pessoas num curto espaço de tempo, existe uma consciência acrescida da importância do planeamento dessas situações.

Com efeito, a gestão e organização de multidões em situações de emergência tornou-se uma importante área de estudo ao longo dos últimos anos e desempenha, hoje, um papel importante no desenho de um edifício ou área.

Dados os desafios - quer de ordem prática, quer de ordem financeira - que a realização de simulacros coloca, é cada vez mais comum o uso de técnicas de simulação para estudar estas situações, existindo diversos tipos de sistemas, como as simulações baseadas na dinâmica de fluxo, as simulações baseadas em autómatos e simulações baseadas em agentes.



\newpage
%----------------------------------------------------------------------------------------
%	EUNCIADO 
%----------------------------------------------------------------------------------------

\section{Enunciado}

\subsection{Descrição}

Ocorreu um incêndio, uma inundação, a libertação de um gás nocivo, um qualquer acidente que obriga à evacuação daqueles presentes num dado local. Esse local possui múltiplas saídas de emergência e também obstáculos. Os indivíduos encontram-se distribuídos pelo local, ocupados nas suas tarefas usuais. Aquando da deteção do acidente, todos os indivíduos procuram atingir uma das saídas de emergência, o mais rapidamente possível.

Alguns agentes poderão ser altruístas, no sentido de ajudarem acidentados a deslocarem-se até à saída, outros poderão simplesmente querer "salvar a pele", exibindo um comportamento mais egoísta, conforme se descreve adiante.


\subsection{Objetivos}

Realizado no âmbito da unidade curricular de Agentes e Inteligência Artificial Distribuída, com este trabalho pretende desenvolver-se um programa que permita simular a interação de agentes confinados a um espaço concreto e limitado, podendo o utilizador configurar o local do acidente, especificando, por um lado, o tipo, número e localização dos agentes a evacuar e, por outro, o número e localização de saídas de emergência e obstáculos.


\subsection{Resultados Esperados e Avaliação}

Experimentando diferentes configurações para o local do acidente, seja variando, por um lado, o tipo, número ou localização dos agentes a evacuar e, por outro, o número e localização de saídas de emergência e obstáculos, será possível observar como estas variações se refletem no tempo médio e máximo de evacuação ou no número de feridos.

\section{Ferramentas}
A implementação do programa descrito será realizada usando a plataforma Repast Symphony + SAJaS.

The Repast has been used extensively in social simulation applications. The latest
version of Repast is Repast Simphony. Models can created with the visual designer
(e.g., visual point-and-click tools are provided for designing agent model, specifying
agent behavior, executing model, and examining results), or written in Java or any
language that runs on the Java virtual machine. For example, users can design the
logical structure, spatial structure (e.g., geographic maps and networks) and behaviors
of their agent models by point-and-click. The simulation is visual and results are
stored. In addition, Repast includes automated results analysis connections to a
variety of spreadsheet, visualization, data mining, and statistical analysis tools 

Para que serve

Descrição das características principais

Realce das funcionalidades relevantes para o trabalho


\section{Especificação}
\subsection{Agentes}

Podem distinguir-se dimensões distintas no comportamento exibido durante uma evacuação: por um lado, o espaço a evacuar e a sua configuração, e, por outro lado, as características psicológicas e sociais que afetam a resposta dos que participam na evacuação.

Assume-se que, em situações de emergência, os indivíduos entram em pânico e ficam, por isso, propensos a tomar decisões irracionais. Mais ainda, as pessoas tentam mover-se tão depressa quanto possível, devendo evitar obstáculos e sofrer ferimentos.

No caso, assumem-se 
In our agent-based models, several types of people are explored (e.g., men, women,
children, security guards and evacuation leaders if necessary), as well as the various
attributes (shown in Table 1) of agents are taken into account. 

The attributes of people and the place configuration can be easily designed in Repast in which person
is regarded as agent and place as “network” and “Grid”. 
In addition, the environmental characteristics that this model considered are as
follows [28]:  
The total number of people in the area
The number of exits
The number of policeman in the area
The number of security guard in the area


Na implementação do projeto deverá ser usada uma arquitetura de Subsunção, em que os comportamentos são definidos como regras, 

\setlength{\tabcolsep}{20pt}
\renewcommand{\arraystretch}{1.3}
\begin{table}[H]
	\centering
	\label{my-label}
	\caption{Atributos dos agentes a implementar.}
	\begin{tabular}{@{}lll@{}}
		\toprule
		\rowcolor[HTML]{FFFFFF} 
	\textbf{Atributos}           & \textbf{Tipo}  & \textbf{Descrição}                                                                                                                                   \\ \midrule
		\rowcolor[HTML]{FFFFFF} 
		idade                & int   & {[}5, 65{]}                                                                                                                                 \\
		\rowcolor[HTML]{FFFFFF} 
		género               & int   & \begin{tabular}[c]{@{}l@{}}0: masculino\\ 1: feminino\end{tabular}                                                                          \\
		\rowcolor[HTML]{FFFFFF} 
		conhecimento da área & float & {[}0, 1{]}                                                                                                                                  \\
		\rowcolor[HTML]{FFFFFF} 
		liderança            & int   & \begin{tabular}[c]{@{}l@{}}0: seguidor\\ 1: líder\end{tabular}                                                                              \\
		\rowcolor[HTML]{FFFFFF} 
		independência        & int   & \begin{tabular}[c]{@{}l@{}}0: segue o líder do grupo\\ 1: não segue o líder\end{tabular}                                                    \\
		\rowcolor[HTML]{FFFFFF} 
		integridade física   & float & \begin{tabular}[c]{@{}l@{}}{[}0, 1{]}\\ 0: morto\\ \textless= 0,4: incapaz de se mover\\ condiciona a velocidade a que se move\end{tabular} \\
		\rowcolor[HTML]{FFFFFF} 
		fadiga               & float & \begin{tabular}[c]{@{}l@{}}{[}0, 1{]}\\ \textless= 0,4: incapaz de se mover\\ condiciona a velocidade a que se move\end{tabular}            \\
		\rowcolor[HTML]{FFFFFF} 
		estado de pânico     & float & \begin{tabular}[c]{@{}l@{}}{[}0,1{]}\\ condiciona a velocidade a que se move\end{tabular}                                                   \\
		\rowcolor[HTML]{FFFFFF} 
		velocidade inicial   & float & condicionada pelo género e da idade                                                                                                         \\
		\rowcolor[HTML]{FFFFFF} 
		velocidade máxima    & float & condicionada pelo género e da idade                                                                                                        
	\end{tabular}

\end{table}


\subsection{Interação}
Protocolos de interação

\subsection{Planeamento}
Para a implementação, importa distinguir a divisão do programa em módulos:
Representação do Espaço;
Representação dos Agentes;
Definição das Interações;

Com base nesta divisão, definiram-se as seguintes etapas:
-Especificação e planeamento;
-Implementação do espaço;
-Implementação dos agentes;
-Implementação das interações entre agentes;
-Exploração de diferentes cenários e recolha e avaliação de métricas.


\section{Conclusão}

Finda esta primeira fase, consideram-se atingidos os objetivos definidos para esta primeira fase: foi feita a descrição do projeto - ... Adicionalmente, após o estudo de diversas ferramentas, definiu-se a combinação Repast+SAJaS como a plataforma multiagente a usar no processo de  desenvolvimento.

Agentes, suas estratégias e interacções
Resultados esperados e como avaliá-los

\section{Recursos}
\subsection{Bibliografia}
[1] Almeida, João; Rosseti, Rosaldo; Coelho, António: Crowd Simulation Modeling Applied to Emergency and Evacuation Simulations using Multi-Agent Systems. (2011)

\subsection{Software}

\end{titlepage}
\end{document}