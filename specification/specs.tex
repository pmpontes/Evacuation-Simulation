%----------------------------------------------------------------------------------------
%	PACKAGES AND OTHER DOCUMENT CONFIGURATIONS
%----------------------------------------------------------------------------------------

\documentclass[12pt]{article}

\usepackage[utf8]{inputenc}
\usepackage[T1]{fontenc}
\usepackage{lmodern}
\usepackage{parselines}
\usepackage[portuguese]{babel}
\usepackage{graphicx}
\usepackage[document]{ragged2e}
\usepackage{listings}
\usepackage{xcolor}
\usepackage{geometry}
\geometry{
	a4paper,
	total={170mm,257mm},
	left=30mm,
	right=30mm,
	top=30mm,
	bottom=30mm,
}
\usepackage{amsmath}
\usepackage{hyperref}
\usepackage{url}
\usepackage{float}
\usepackage{tabularx}
\usepackage{booktabs}
\usepackage{indentfirst}
\usepackage{colortbl}
\usepackage{caption} 
\usepackage{enumitem}
\captionsetup[table]{skip=10pt}

\graphicspath{ {images/} }

\colorlet{punct}{red!60!black}
\definecolor{background}{HTML}{EEEEEE}
\definecolor{delim}{RGB}{20,105,176}
\colorlet{numb}{magenta!60!black}


\lstdefinelanguage{json}{
    basicstyle=\normalfont\ttfamily,
    numbers=left,
    numberstyle=\scriptsize,
    stepnumber=1,
    numbersep=8pt,
    showstringspaces=false,
    breaklines=true,
    frame=lines,
    backgroundcolor=\color{background},
    literate=
     *{0}{{{\color{numb}0}}}{1}
      {1}{{{\color{numb}1}}}{1}
      {2}{{{\color{numb}2}}}{1}
      {3}{{{\color{numb}3}}}{1}
      {4}{{{\color{numb}4}}}{1}
      {5}{{{\color{numb}5}}}{1}
      {6}{{{\color{numb}6}}}{1}
      {7}{{{\color{numb}7}}}{1}
      {8}{{{\color{numb}8}}}{1}
      {9}{{{\color{numb}9}}}{1}
      {:}{{{\color{punct}{:}}}}{1}
      {,}{{{\color{punct}{,}}}}{1}
      {\{}{{{\color{delim}{\{}}}}{1}
      {\}}{{{\color{delim}{\}}}}}{1}
      {[}{{{\color{delim}{[}}}}{1}
      {]}{{{\color{delim}{]}}}}{1},
}

\definecolor{dkgreen}{rgb}{0,0.6,0}
\definecolor{gray}{rgb}{0.5,0.5,0.5}
\definecolor{mauve}{rgb}{0.58,0,0.82}

\lstset{frame=tb,
  language=Java,
  aboveskip=3mm,
  belowskip=3mm,
  showstringspaces=false,
  columns=flexible,
  basicstyle={\small\ttfamily},
  numbers=none,
  numberstyle=\tiny\color{gray},
  keywordstyle=\color{blue},
  commentstyle=\color{dkgreen},
  stringstyle=\color{mauve},
  breaklines=true,
  breakatwhitespace=true,
  tabsize=3
}

\hypersetup{
  colorlinks, linkcolor=black
}

\begin{document}

\begin{titlepage}

\newcommand{\HRule}{\rule{\linewidth}{1mm}} % Defines a new command for the horizontal lines, change thickness here

\center % Center everything on the page
 
%----------------------------------------------------------------------------------------
%	HEADING SECTIONS
%----------------------------------------------------------------------------------------

\includegraphics{feup.jpg}

\textsc{\large Agentes e Inteligência Artificial Distribuída}\\[0.8cm] % Major heading such as course name
\textsc{\large 4º ano do Mestrado Integrado em Engenharia Informática e Computação}\\[0.8cm] % Minor heading such as course title

%----------------------------------------------------------------------------------------
%	TITLE SECTION
%----------------------------------------------------------------------------------------

\HRule \\[1.2cm]
{ \huge \bfseries \textit{Simulação de Evacuação com Agentes}}\\[0.6cm] % Title of your document
{ \large \bfseries Relatório Intercalar} \\[0.6cm]
\HRule \\[2cm]
 
%----------------------------------------------------------------------------------------
%	AUTHOR SECTION
%----------------------------------------------------------------------------------------


% If you don't want a supervisor, uncomment the two lines below and remove the section above
\Large \emph{Estudantes:}\\[0.5cm] \normalsize
Gil \textsc{Domingues}\\[0.1cm]  
- up201304646@fe.up.pt\\[0.1cm]
Pedro \textsc{Pontes}\\[0.1cm]
- up201305367@fe.up.pt\\[2cm] 

%----------------------------------------------------------------------------------------
%	DATE SECTION
%----------------------------------------------------------------------------------------

{\large \today}\\[0cm] % Date, change the \today to a set date if you want to be precise

%----------------------------------------------------------------------------------------
%	TABLE OF CONTENTS & LISTS OF FIGURES AND TABLES
%----------------------------------------------------------------------------------------

\clearpage 

\tableofcontents

\clearpage 
%----------------------------------------------------------------------------------------
%	INTRODUÇÃO
%----------------------------------------------------------------------------------------
\justify\normalsize

\section{Introdução} 

Uma evacuação implica mover pessoas de um dado local devido à ocorrência de uma situação de (potencial) catástrofe. Exemplos incluem a evacuação de um edifício em chamas ou de uma localidade, antes, durante ou após um desastre natural, como uma cheia ou terramoto. 

Evacuar grandes multidões é um desafio, independentemente das circunstâncias. Tipicamente, de uma evacuação de emergência resultam feridos - ou mesmo mortes -, devido ao caos e pânico que se geram.

Com o aumento da frequência de situações que implicam a evacuação de um elevado número de pessoas num curto espaço de tempo, existe uma consciência acrescida da importância do planeamento dessas situações.

Com efeito, a gestão e organização de multidões em situações de emergência tornou-se uma importante área de estudo ao longo dos últimos anos e desempenha, hoje, um papel importante no planeamento de um edifício ou área.

Dados os desafios - quer de ordem prática, quer de ordem financeira - que a realização de simulacros coloca, é cada vez mais comum o uso de técnicas de simulação para estudar estas situações. De facto, existem já diversos tipos de sistemas, como as simulações baseadas na dinâmica de fluídos, as simulações baseadas em autómatos e as simulações baseadas em agentes.

%----------------------------------------------------------------------------------------
%	EUNCIADO 
%----------------------------------------------------------------------------------------

\section{Enunciado}

\subsection{Descrição}

Ocorreu um incêndio, uma inundação, a libertação de um gás nocivo, um qualquer acidente que obriga à evacuação daqueles presentes num dado local. Esse local possui múltiplas saídas de emergência e também obstáculos. Os indivíduos encontram-se distribuídos pelo local, ocupados nas suas tarefas usuais. Aquando da deteção do acidente, todos os indivíduos procuram atingir uma das saídas de emergência, o mais rapidamente possível.

Alguns agentes poderão ser altruístas, no sentido de ajudarem acidentados a deslocarem-se até à saída, outros poderão simplesmente querer «salvar a pele», exibindo um comportamento mais egoísta, conforme se descreve adiante.


\subsection{Objetivos}

Realizado no âmbito da unidade curricular de Agentes e Inteligência Artificial Distribuída, com este projeto pretende desenvolver-se um programa que permita simular a interação de agentes confinados a um espaço concreto e limitado perante a necessidade de evacuar esse espaço, podendo o utilizador definir diferentes cenários, especificando, por um lado, o tipo, número e localização dos agentes a evacuar e, por outro, o número e localização de saídas de emergência e obstáculos.


\subsection{Resultados Esperados}

Como mencionado, será possível - e relevante - avaliar diferentes cenários, através da experimentação com:
\begin{itemize}
	\item diferentes configurações para o local do acidente, variando o número e localização de saídas de emergência e obstáculos;
	\item diferentes combinações de agentes a evacuar, variando o seu tipo, número ou localização.
\end{itemize}

Deste modo, será possível observar-se como estas variações se refletem em métricas como o tempo médio e máximo de evacuação ou o número de feridos.

\section{Ferramentas}
A implementação do programa descrito será realizada usando \textit{Repast}, uma \textit{framework open-source} que permite criar, analisar e experimentar com mundos artificiais populados por agentes que interagem de forma não trivial.

Concretamente, irá utilizar-se a sua mais recente versão - \textit{Repast Simphony} -, no \textit{flavour RepastJ}, que permite programar em \textit{Java} a estrutura espacial, a estrutura lógica e os comportamentos dos agentes.

Tendo sido amplamente utilizado em aplicações de simulação, considera-se de particular utilidade, por um lado, o foco em modelar o comportamento social e, por outro, a recolha de métricas associadas às simulações realizadas. Por último, tem-se a vantagem de poder acompanhar, de forma visual, o decorrer da simulação.

Adicionalmente, irá utilizar-se a \textit{API SAJaS}, com vista a facilitar o desenvolvimento deste sistema multiagente, dado oferecer funcionalidades \textit{JADE}. 
No caso, as funcionalidades de maior interesse serão as capacidades de comunicação entre agentes, visando simular as interações expectáveis num cenário de evacuação.

\section{Especificação}
\subsection{Agentes}

Podem distinguir-se dimensões distintas no comportamento exibido durante uma evacuação: por um lado, o espaço a evacuar e a sua configuração, e, por outro lado, as características psicológicas e sociais que afetam a resposta dos que participam na evacuação.

Assume-se que, em situações de emergência, os indivíduos entram em pânico e ficam, por isso, propensos a tomar decisões irracionais. Mais ainda, as pessoas tentam mover-se tão depressa quanto possível, devendo evitar obstáculos e ferimentos.
Deste modo, tem-se que os agentes a implementar serão autónomos, proativos e reativos.

No caso, os modelo definidos assumem a existência de homens, mulheres e crianças, caracterizados por diversos atributos, conforme definido na Tabela 1.

\setlength{\tabcolsep}{20pt}
\renewcommand{\arraystretch}{1.3}
\begin{table}[H]
	\centering
	\label{agent-attributes}
	\caption{Atributos dos agentes a implementar.}
	\begin{tabular}{@{}lll@{}}
		\toprule
		\rowcolor[HTML]{FFFFFF} 
	\textbf{Atributos}           & \textbf{Tipo}  & \textbf{Descrição}                                                                                                                                   \\ \midrule
		\rowcolor[HTML]{FFFFFF} 
		idade                & int   & {[}5, 65{]}                                                                                                                                 \\
		\rowcolor[HTML]{FFFFFF} 
		género               & int   & \begin{tabular}[c]{@{}l@{}}0: masculino\\ 1: feminino\end{tabular}                                                                          \\
		\rowcolor[HTML]{FFFFFF} 
		conhecimento da área & float & \begin{tabular}[c]{@{}l@{}}{[}0, 1{]} \\probabilidade de seguir um caminho\\ até à saída mais próxima\\\end{tabular}                                                                                                                                     \\
		\rowcolor[HTML]{FFFFFF} 
		independência        & float   & \begin{tabular}[c]{@{}l@{}}{[}0, 1{]} \\probabilidade de seguir (ou não) outros\\\end{tabular}  \\ 
		\rowcolor[HTML]{FFFFFF} 
		altruísmo   & float & \begin{tabular}[c]{@{}l@{}}{[}0, 1{]} \\probabilidade de ajudar outros\\\end{tabular}    \\                                                  
		\rowcolor[HTML]{FFFFFF} 
		mobilidade   & float & \begin{tabular}[c]{@{}l@{}}{[}0, 1{]}\\ condiciona a velocidade a que se move\end{tabular} \\
		\rowcolor[HTML]{FFFFFF} 
		fadiga               & float & \begin{tabular}[c]{@{}l@{}}{[}0, 1{]}\\ \textgreater 0,8: incapaz de se mover\\ condiciona a velocidade a que se move\end{tabular}            \\
		\rowcolor[HTML]{FFFFFF} 
		estado de pânico     & float & \begin{tabular}[c]{@{}l@{}}{[}0,1{]}\\ condiciona a velocidade a que se move\end{tabular}                                                   \\
		\rowcolor[HTML]{FFFFFF} 
		velocidade inicial   & float & condicionada pelo género e pela idade                                                                                                         \\
		\rowcolor[HTML]{FFFFFF} 
		velocidade máxima    & float & condicionada pelo género e pela idade                                                                                                        
	\end{tabular}

\end{table}


\subsection{Interações}

Com vista a simular de forma mais fidedigna as condições de uma evacuação de emergência, prevê-se a implementação das seguintes interações entre agentes:

\begin{itemize}
	
\item Empurrar;
	
Uma pessoa em pânico pode empurrar uma outra no seu caminho, derrubando-a. Uma
pessoa que seja derrubada verá a sua integridade física e, consequentemente, a sua mobilidade, reduzidas.

\item Ajudar;

	Uma pessoa pode ajudar outra, guiando-a até à saída, caso em que a mobilidade da pessoa altruísta fica diminuída (cansa-se mais rapidamente).

\item Gritar.
	O facto de uma pessoa gritar pode aumentar o estado de pânico das pessoas em redor.
\end{itemize}

Além destas interações, deverá ser implementado outro tipo de interação, por modo a permitir a troca de informações entre as pessoas - nomeadamente, no que respeita à localização das saídas.

\subsection{Planeamento}
Para a implementação, definiram-se as seguintes etapas:
\begin{enumerate}
	\item Especificação e planeamento;
	\item Implementação de:
	 \begin{enumerate} 
	 	\item Agente;
	 	\item Espaço;
	 	
	 	Teste e análise do comportamento de um agente num espaço.
		\item Interação entre agentes;
		
		Teste e análise do comportamento de vários agentes num espaço. 
	\end{enumerate}
	\item Exploração de diferentes cenários e recolha e avaliação de métricas.
\end{enumerate}


\section{Conclusão}

No final, consideram-se atingidos os objetivos definidos para esta primeira fase: foi feita a descrição do projeto e do seu objetivo - desenvolver um programa que permita simular a interação de agentes confinados a um espaço concreto e limitado perante a necessidade de evacuar esse espaço.

Após o estudo de diversas ferramentas, definiu-se a combinação \textit{Repast} e \textit{SAJaS} como a plataforma multiagente a utilizar no processo de  desenvolvimento.

Caracterizaram-se os modelos de agente a implementar e definiram-se as interações entre eles.

Iniciou-se, ainda, a fase de implementação, com uma primeira definição dos agentes e espaço a evacuar.



\section{Recursos}
\subsection{Bibliografia}
[1] Almeida, João; Rosseti, Rosaldo; Coelho, António: \textit{Crowd Simulation Modeling Applied to Emergency and Evacuation Simulations using Multi-Agent Systems}. (2011)

\subsection{Software}
[1] \textit{Plugin Repast Simphony} para \textit{Eclipse IDE};

[2] \textit{SAJaS}.

\end{titlepage}
\end{document}